\documentclass[aspectratio=169]{beamer}
\usepackage[T1]{fontenc}

%%%%%%%%%%%%%%%%%%%%%%%%%%%%%%%%%%%%%%%%%%%%%%%%%%%%%%%%%
%                     STYLE OPTIONS                     %
%%%%%%%%%%%%%%%%%%%%%%%%%%%%%%%%%%%%%%%%%%%%%%%%%%%%%%%%%

\usetheme[dunkelblau,
		  redhattext=true,
		  %displaynavigation,compress,
		  %displayinstitute,  
		  ownlogo=true,  
		  sponsorlogo=true,         
		  %displayframetotal
		  ]{rptu}
		  
% Read the README, some options need specific compiler
% options, only installing the RedHatText font has some 
% pitfalls

%%%%%%%%%%%%%%%%%%%%%%%%%%%%%%%%%%%%%%%%%%%%%%%%%%%%%%%%%
%                        META DATA                      %
%%%%%%%%%%%%%%%%%%%%%%%%%%%%%%%%%%%%%%%%%%%%%%%%%%%%%%%%%

\title[RPTU \LaTeX\ Presentation]{A first RPTU \LaTeX\ Presentation: \\ all theme options}
\subtitle{with the new RPTU Kaiserslautern-Landau Corporate Design}
\date{Kaiserslautern, \today}
\author[Last Name]{Firstname Lastname}
\institute[Short Institute]{Institute}

% info for thank you page
\thankstitle{Thank you title}
\thankssubtitle{content of thanks subtitle}
\thanksinfo{and other info as needed}


% fill this command and set ownlogo=true in \usetheme 
% make sure to specify the width/height in \includegraphics
% this will override the default RPTU on the title page
\renewcommand{\ownlogo}{\includegraphics[width=6cm, height=2.5cm]{example-image-a}}

% fill this command and set sponsorlogo=true in \usetheme 
% make sure to specify the width/height in \includegraphics
% this will be placed in the upper right corner of the slide
\renewcommand{\sponsorlogo}{\includegraphics[width=3cm, height=2.5cm]{example-image-b}}


%%%%%%%%%%%%%%%%%%%%%%%%%%%%%%%%%%%%%%%%%%%%%%%%%%%%%%%%%
%                CONTENT OF PRESENTATION                %
%%%%%%%%%%%%%%%%%%%%%%%%%%%%%%%%%%%%%%%%%%%%%%%%%%%%%%%%%

\usepackage{booktabs} % nicer tables
\usepackage{metalogo} % typesetting XeLaTeX and LuaLaTex logos

\begin{document}
\begin{frame}
	\titlepage
\end{frame}

\section{Theme Options}
\rptusectionpage

\begin{frame}{Available options~I - Colors and RPTU Hausschrift (Red Hat Text)}
	\framesubtitle{\texttt{\textbackslash usetheme[options]\{rptu\}}}
	\begin{tabular}{ll}
		\textbf{Option} & \textbf{Description}\\ \toprule
		\texttt{dunkelblau|hellblau} & Set color scheme of presentation\\
		\texttt{rot|orange} & (\textbf{dunkelblau} is default)\\
		\texttt{dunkelgruen|hellgruen} & Each line defines a colorscheme,\\
		\texttt{blaugrau|gruengrau} & you specify the main color and the\\ 
		\texttt{violett|pink} & secondary color is set automatically\\ \midrule		\texttt{redhattext=true|false} & Use/\textbf{Do not use} the Red Hat Text Font \\ & needs specific compilers $\Rightarrow$ read remarks (Slide~\pageref{remarks-font}) \\\bottomrule
	\end{tabular}
	\vspace*{2ex}
	
	Defaults are the descriptions written in \textbf{bold}.
\end{frame}


\begin{frame}{Using Red Hat Text}
	\label{remarks-font}
	\begin{itemize}
		\item Make sure that the static versions of the font are installed on your computer
		\item We use the following \texttt{.ttf} files \begin{itemize}
			\item RedHatText-Regular as normal text
			\item RedHatText-Italic as \textit{italic} \texttt{\textbackslash textit}
			\item RedHatText-SemiBold as \textbf{bold} \texttt{\textbackslash textbf}
			\item RedHatText-SemiBoldItalic for \textbf{\textit{bold and italic}}
			\item RedHatText-Bold for {\fontseries{k}\selectfont title and the sectionpage}, can be accessed with \texttt{\textbackslash fontseries\{k\}\textbackslash selectfont} 
		\end{itemize}
		\item We use the package \texttt{fontspec} for setting this font $\Rightarrow$ you need to compile with \XeLaTeX\ (XeLaTeX) or Lua\LaTeX\ (LuaLaTeX) if you want to use Red Hat Text
		\item If you use \XeLaTeX\ or Lua\LaTeX\ with the option \texttt{redhattext=false} then Arial is set as main font family with Arial Black as substitute for {\fontseries{k}\selectfont RedHatText-Bold} (you need to install Arial/Arial Black manually under Ubuntu in that case)
		\item With pdfLaTeX the main font family is Computer Modern (\LaTeX\ default)
	\end{itemize}
\end{frame}


\begin{frame}{Available options~II - Logos}
	\framesubtitle{\texttt{\textbackslash usetheme[options]\{rptu\}}}
	\begin{tabular}{ll}
		\textbf{Option} & \textbf{Description}\\ \toprule
		\texttt{ownlogo=true|false} & Use/\textbf{Do not use} a user specified logo \\ 
		& (instead of default RPTU logo) \\
		\texttt{sponsorlogo=true|false} & Use/\textbf{Do not use} a user specified sponsor logo \\ & (placed in upper right corner) \\\bottomrule
	\end{tabular}
	\vspace*{2ex}
	
	Defaults are the descriptions written in \textbf{bold}.\\ In both cases you need to specify the exact image with dimensions by filling 
	\begin{itemize}
		\item \texttt{\textbackslash renewcommand{\textbackslash ownlogo}\{\}} \hspace{1em}$\Rightarrow$\hspace{1em} image A on the title page
		\item \texttt{\textbackslash renewcommand{\textbackslash sponsorlogo}\{\}}  \hspace{1em}$\Rightarrow$\hspace{1em} image B on the title page
	\end{itemize}
	with e.g. \texttt{\textbackslash includegraphics[width=3cm]\{your\_file.png\}}
\end{frame}

\begin{frame}{Available options~III - Footline}
	\framesubtitle{\texttt{\textbackslash usetheme[options]\{rptu\}}}
	\begin{tabular}{ll}
		\textbf{Option} & \textbf{Description}\\ \toprule
		\texttt{institute=true|false} &Show/\textbf{Hide} short institute in footline \\
		\texttt{displayinstitute} & Show short institute in footline \\ 
		\texttt{hideinstitute} & Hide short institute in footline \\ \midrule
		\texttt{frametotal=true|false} &Show/\textbf{Hide} total number of slides in footline\\
		\texttt{displayframetotal} & Show total number of slide in footline\\ 
		\texttt{hideframetotal} & Hide total number of slide in footline\\  \bottomrule
	\end{tabular}
	\vspace*{2ex}
	
	Defaults are the descriptions written in \textbf{bold}.
	
	This version uses the defaults, i.e. no institute in footline, no total number of frames.
\end{frame}

\begin{frame}{Available options~IV - Navigation}
	\framesubtitle{\texttt{\textbackslash usetheme[options]\{rptu\}}}
	\begin{tabular}{ll}
		\textbf{Option} & \textbf{Description}\\ \toprule
		\texttt{navigation=true|false} &Show/\textbf{Hide} navigation in headline \\
		\texttt{displaynavigation} & Show navigation in headline \\ 
		\texttt{hidenavigation} & Hide navigation in headline \\  \midrule
		\texttt{compress} & equivalent to beamer's compress \\\bottomrule
	\end{tabular}
	\vspace*{2ex}
	
	Defaults are the descriptions written in \textbf{bold}. If you do not specify any navigation option, no navigation bar with miniframes will be shown.
	
	This presentation uses the defaults so no navigation bar.
\end{frame}


\begin{frame}{Options}{Some Remarks}
	\begin{itemize}
		\item Use \texttt{navigation=true} in combination with the option \texttt{compress} to obtain a single line for the navigation symbols.
		\item If \texttt{institute=true} the \texttt{\textbackslash institute} is used to set the name of department/institute/affiliation both in the footline and on the title page\\
		\texttt{\textbackslash institute[name in footline]\{name on title page\}}
		\item Use \texttt{\textbackslash documentclass[aspectratio=169]\{beamer\}} for getting a 16:9 aspect ratio instead of 4:3
	\end{itemize}
\end{frame}

\rptuthankyou %create thank you page

\end{document}
